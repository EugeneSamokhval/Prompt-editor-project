\sectioncentered*{Реферат}
\thispagestyle{empty}

\MakeUppercase{Платформа для интерактивного формирования запросов к языковым и генеративным нейросетям}: дипломный проект/ Е.~С. Самохвал. -- Минск : БГУИР, \the\year{}. -- п.з. -- ~\pageref*{LastPage}~с., чертежей (плакатов) -- 6~л. формата~А1.

\vspace{4\parsep} 

Дипломный проект выполнен на 6 листах А1 с пояснительной запиской на~\pageref*{LastPage} страницах, без приложений справочного или информационного характера. Пояснительная записка включает 4~раздела, 27~рисунков, 3~таблицы, 9~формул, 29~литературных источников.

Целью дипломного проекта является создание веб-платформы, обеспечивающей интерактивное формирование, структурирование и оптимизацию запросов к языковым и генеративным нейросетям с возможностью последующей интеграции в корпоративные и образовательные процессы.

Предметом исследования является архитектура, методы оптимизации и интерфейсы систем интерактивного формирования запросов к LLM и генеративным моделям.

Данный программный модуль позволяет создавать качественные, структурированные и адаптируемые запросы к нейросетям с учетом контекста, цели и формата ответа.

В первом разделе пояснительной записки проведен анализ существующих платформ и методик оптимизации запросов, сформулированы критерии качества и предложен сравнительный обзор.

Во втором разделе спроектирована модель системы, включая архитектуру платформы, логические и физические модели, информационные потоки и безопасность.

В третьем разделе реализованы клиентская часть на Vue 3, серверная логика на FastAPI, а также интеграция с API FusionBrain и DeepSeek, проведена валидация и тестирование.

В четвертом разделе приведено технико-экономическое обоснование разрабатываемого программного продукта, включая расчет затрат и оценку эффективности (NPV, ROI, PI).

Результатом дипломного проектирования является программа, обеспечивающая пользовательски-ориентированный интерфейс и высокое качество взаимодействия с языковыми моделями, реализованная по принципам модульности и безопасности.


\clearpage