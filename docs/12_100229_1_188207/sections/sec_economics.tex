\section{Экономическое обоснование разработки платформы для интерактивного формирования запросов к языковым и генеративным нейросетям}
\label{sec:economics}

\subsection{Характеристика разработанного по индивидуальному заказу  программного средства}

Разрабатываемая интеллектуальная система машинного перевода предназначена для автоматизированного перевода текстов на естественных языках с учетом контекста, стилистики и грамматических особенностей. Её основная цель — обеспечение точного и качественного перевода, особенно в специализированных областях, таких как техническая документация, медицина, юриспруденция и наука, где важно не только передать общий смысл, но и сохранить правильную терминологию.  

Система сочетает современные методы обработки естественного языка с архитектурой трансформеров, что позволяет учитывать широкий контекст текста и адаптироваться к различным стилям речи. В отличие от традиционных статистических моделей, разработанный инструмент ориентирован на глубокое понимание текста, что улучшает точность перевода и снижает количество ошибок. В ходе эксплуатации система может совершенствоваться, адаптируясь к особенностям используемых текстов, что особенно важно в профессиональной среде.

Предусмотрена возможность взаимодействия с системой на разных уровнях, позволяющая учитывать индивидуальные потребности пользователей. Это создаёт условия для её дальнейшего развития и повышения точности перевода в конкретных областях.

Программное средство предназначено для использования как отдельными пользователями, так и корпоративными клиентами, которым необходимо быстро и точно переводить большие объемы данных. Благодаря модульной архитектуре и поддержке API система легко интегрируется в бизнес-процессы, системы документооборота, онлайн-сервисы и мобильные приложения.  

Разработка ведется в ответ на растущую потребность в высокоточном машинном переводе, который мог бы заменить или дополнить работу профессиональных переводчиков, сокращая временные и финансовые затраты. Ожидаемый экономический эффект заключается в автоматизации процессов перевода, снижении расходов на локализацию контента и ускорении обмена информацией между международными партнерами, что делает систему полезной как для коммерческих организаций, так и для научного сообщества.

\subsection{Расчет затрат на разработку и цены программного средства по индивидуальному заказу}

Цена программного средства определена на основе полных затрат на разработку программного средства и включает в себя следующие статьи затрат: 
\begin{itemize}
    \item затраты на основную заработную плату разработчиков;
    \item затраты на дополнительную заработную плату разработчиков;
    \item отчисления на социальные нужды;
    \item прочие затраты (амортизационные отчисления, расходы на электроэнергию, командировочные расходы, арендная плата за офисные помещения и оборудование, расходы на управление и реализацию и т.п.);
    \item общая сумма затрат на разработку;
    \item плановая прибыль, включаемая в цену программного средства;
    \item отпускная цена программного средства;
\end{itemize}

1. Затраты на основную заработную плату команды разработчиков.
Основная заработная плата исполнителей проекта определяется по формуле:

\begin{equation}
    \mbox{З}_{\mbox{о}}=\mbox{К}_{\mbox{пр}}\cdot \sum_{i=1}^{n}{\mbox{З}_{\mbox{ч}i}\cdot t_{i}},
\end{equation}

где	$n$ -- количество исполнителей, занятых разработкой конкретного ПО;

    $\mbox{К}_{\mbox{пр}}$ -- коэффициент премий (1,5); 
    
    $\mbox{З}_{\mbox{ч}i}$ -- часовая заработная плата i-го исполнителя (руб.); 
    
    $t_{i}$ -- трудоемкость работ, выполняемых i-м исполнителем (ч).
    

Разработкой программного средства занимались следующие лица: бизнес–аналитик, два программиста, тестировщик, дизайнер, переводчик. Часовая заработная плата каждого исполнителя определялась путем деления его месячной заработной платы (оклад) на количество рабочих часов в месяце.

Количество рабочих часов в месяце составляет 168.

Расчет основной заработной платы представлен в таблице \ref{tab1}.

\begin{table}[!h!t]
\caption{Расчет основной заработной платы }
\label{tab1}
\centering

	\begin{tabular}{| >{\raggedright}m{0.02\textwidth}
			| >{\centering\arraybackslash}m{0.22\textwidth}
			| >{\centering\arraybackslash}m{0.15\textwidth}  
			| >{\centering\arraybackslash}m{0.13\textwidth}
			| >{\centering\arraybackslash}m{0.2\textwidth}
			| >{\centering\arraybackslash}m{0.12\textwidth}|}

\hline
№ & Участник команды & Месячный оклад, р. & Часовой оклад, р. & Трудоемкость работ, ч. & Итого, р. \\ 

\hline
1 & 2 & 3 & 4 & 5 & 6   \\ 

\hline
1 & Бизнес–аналитик & 3140  & 18,69 & 50  & 934,50 \\    

\hline
2 & Программист & 3440  & 20,48 & 725 & 29696,00 \\ 

\hline
3 & Тестировщик & 2520  & 15,00 & 200 & 3000,00 \\

\hline
4 & Дизайнер & 2700  & 16,07 & 72  & 1157,04\\

\hline
5 & Переводчик & 2120  & 12,62 & 800 & 10096,00 \\

\hline
\multicolumn{5}{|l|}{Итого}    & 44883,54   \\ 

\hline
\multicolumn{5}{|l|}{Премия(50\%)}    & 22441,77   \\ 
\hline

\multicolumn{5}{|l|}{\begin{tabular}[c]{@{}l@{}}Итого затраты на основную заработную плату\\ разработчиков\end{tabular}}  & 67325,31   \\ 
\hline
\end{tabular}
\end{table}

2. Затраты на дополнительную заработную плату команды разработчиков включает выплаты, предусмотренные законодательством о труде (оплата трудовых отпусков, льготных часов, времени выполнения государственных обязанностей и других выплат, не связанных с основной деятельностью исполнителей), и определяется по формуле:

\begin{equation}
    \text{З}_{\text{д}} = \frac{\text{З}_{o}\cdot H_\text{д}}{100\%},
\end{equation}


где $H_\text{д}$ -- норматив дополнительной              заработной платы(20 \%);

   $\text{З}_{\text{o}}$ -- затраты на основную заработную плату, (р.);




Дополнительная заработная плата составит:

$$
\text{З}_{\text{д}} = \frac{67325,31 \cdot 20\%}{100\%} = 13465,06 \text{ р}.
$$

3. Отчисления в фонд социальной защиты и обязательного страхования (в фонд социальной защиты населения и на обязательное страхование) определяются в соответствии с действующими законодательными актами по формуле: 

\begin{equation}
    \text{Р}_{\text{соц}} = \frac{(\text{З}_{o} + \text{З}_{\text{д}})\cdot H_\text{соц}}{100\%},
\end{equation}
 
где $H_\text{соц}$ -- норматив отчислений в фонд социальной защиты населения и на обязательное страхование (34,6 \%).
 
$$
 \text{Р}_{\text{соц}} = \frac{(67325,31 + 13465,06)\cdot 34,6\%}{100\%} = 27953,47 \text{ р}.
$$

4. Прочие затраты включаются в себестоимость разработки программного обеспечения в процентах от затрат на основную заработную плату команды разработчиков (табл.2.1) по формуле:

\begin{equation}
    \text{З}_{\text{пр}} = \frac{\text{З}_{o} \cdot \text{Н}_{\text{пр}}}{100},
\end{equation}

где $\text{Н}_{\text{пр}}$ -- норматив прочих затрат (40 \%).

$$
 \text{З}_{\text{пр}} = \frac{67325,31 \cdot 40\%}{100\%} = 26930,12 \text{ р}.
$$

4. Общая сумма затрат на разработку рассчитывается путем суммирования основной заработной платы,
дополнительной заработной платы, отчислений на социальные нужды, прочих затрат. Формула расчета имеет следующий вид:

$$
 \text{З}_{\text{р}} =\text{З}_{\text{о}} + \text{З}_{\text{д}} + \text{Р}_{\text{соц}} + \text{З}_{\text{пр}}
$$

$$
 \text{З}_{\text{р}} = 67325,31+13465,06+27953,47+26930,12 = 135673,96 \text{ р}.
$$

5. Плановая прибыль включает в себя два ключевых компонента: затраты на разработку и рентабельность этих затрат. Рентабельность затрат отражает желаемую прибыльность инвестиций и показывает, какой процент от затрат на разработку составит прибыль. Формула расчета имеет следующий вид:

\begin{equation}
    \text{П}_{\text{п.с}} = \frac{\text{З}_{\text{р}} \cdot \text{Р}_{\text{п.с}}}{100\%},
\end{equation}

где $\text{Р}_{\text{п.с}}$ -- рентабельность затрат на разработку программного средства (25 \%).

$$
 \text{П}_{\text{п.с}} = \frac{135673,96 \cdot 25\%}{100\%} = 33918,49 \text{ р}.
$$

6. Отпускная цена программного средства — это стоимость, по которой продукт предлагается заказчику. Она включает в себя все затраты на разработку, маркетинг и поддержку, а также предполагаемую прибыль. Отпускная цена может варьироваться в зависимости от рыночных условий, конкуренции и уникальных характеристик программного обеспечения. Формула расчета имеет следующий вид:

\begin{equation}
    \text{Ц}_{\text{п.с}} = \text{З}_{\text{р}} + \text{П}_{\text{п.с}}
\end{equation}

$$
 \text{Ц}_{\text{п.с}} = 135673,96 + 33918,49 = 169592,45 \text{ р}.
$$


Результаты расчета затрат на разработку представлены в таблице \ref{tab2}.

\begin{table}[!h!t]
\centering
\caption{Затраты на разработку программного обеспечения}
\label{tab2}
\begin{tabular}{|l|c|}
\hline
\multicolumn{1}{|c|}{Статья затрат}                                                                       & Сумма, руб. \\ \hline
Основная заработная плата команды разработчиков                                                           & 67325,31      \\ \hline
\begin{tabular}[c]{@{}l@{}}Дополнительная заработная плата команды\\ разработчиков\end{tabular}           & 13465,06      \\ \hline
\begin{tabular}[c]{@{}l@{}}Отчисления в фонд социальной защиты и\\ обязательного страхования\end{tabular} & 27953,47      \\ \hline
Прочие затраты                                                                                            & 26930,12      \\ \hline
Общая сумма затрат на разработку                                                                          & 135673,96    \\ \hline
\begin{tabular}[c]{@{}l@{}}Плановая прибыль, включаемая в цену \\программного средства\end{tabular}  &
33918,49 \\ \hline
Отпускная цена программного средства &
169592,45 \\ \hline
\end{tabular}
\end{table}

\subsection{Расчет результата от разработки и использования программного средства по индивидуальному заказу}

Экономический эффект от разработки программного средства по индивидуальному заказу рассчитан для организации-разработчика (резидент Парка высоких технологий) и для организации-заказчика.

1. Для организации-разработчика экономическим эффектом является прирост чистой прибыли, полученной от разработки и реализации программного средства заказчику. Формула расчета имеет следующий вид:

$$
 \text{$\Delta$П}_{\text{ч}} = \text{П}_{\text{п.с}} \cdot (1 - \frac{\text{Н}_{\text{п}} }{100\%}),
$$

где $\text{Н}_{\text{п}}$ -- ставка налога на прибыль, (\%);

   $\text{П}_{\text{п.с}}$ -- прибыль, включаемая в цену программного средства, (р.);

Так как организация-разработчик является резидентом Парка высоких технологий, она освобождена от уплаты налога на прибыль и в формулах $\text{Н}_{\text{п}}=0\%$.

$$
 \text{$\Delta$П}_{\text{ч}} = 33918,49 \cdot (1 - \frac{0\%}{100\%}) = 33918,49 \text{ р}.
$$

Итоговый результат разработки и использования программного средства по индивидуальному заказу составляет 33 918,49 \text{ р}.

Для организации-заказчика расчет экономического эффекта от применения программного обеспечения, разработанного по индивидуальному заказу сторонней организации, выполняется по следующей методике:

1. Экономия на заработной плате и начислениях на заработную плату сотрудников за счет снижения трудоемкости работ считается по формуле:


\begin{equation}
    \text{Э}_{\text{з.п}} = \text{К}_{\text{п.р}}\cdot (t^{\text{без п.с}}_{\text{р}}-t^{\text{с п.с}}_{\text{р}})\cdot \text{Т}_{\text{ч}}\cdot N_{\text{П}}\cdot  (1+\frac{\text{Н}_{\text{д}}}{100\%})\cdot (1+\frac{\text{Н}_{\text{соц}}}{100\%}), 
\end{equation}

где $N_{\text{П}}$ -- плановый объем работ; 

    $t^{\text{без п.с}}_{\text{р}}-t^{\text{с п.с}}_{\text{р}}$ -- трудоемкость выполнения работы до и после внедрения программного продукта, нормо-час; 
    
    $\text{Т}_{\text{ч}}$ -- часовая тарифная ставка, соответствующая разряду выполняемых работ, руб./ч (15 руб./ч.); 
    
    $\text{К}_{\text{пр}}$ -- коэффициент премий, (1,5); 
    
    $\text{Н}_{\text{д}}$ -- норматив дополнительной заработной платы, (20\%); 
    
    $\text{Н}_{\text{соц}}$ -- ставка отчислений от заработной платы, включаемых в себестоимость, (34,6\%).
 
$$
    \text{Э}_{\text{З}} = 1,5 \cdot (8-2)\cdot 15\cdot 168 \cdot  (1+\frac{20\%}{100\%})\cdot (1+\frac{34,6\%}{100\%}) =  36632,74 \text{ р}.
$$
2. Экономия на заработной плате и начислениях на заработную плату в результате сокращения численности работников рассчитывается по формуле:

\begin{equation}
    \text{Э}_{\text{з.п}} = \sum_{i=1}^{n}{\text{$\Delta$Ч}_{\text{i}}\cdot \text{З}_{\text{i}}\cdot  (1+\frac{\text{Н}_{\text{д}}}{100\%})\cdot (1+\frac{\text{Н}_{\text{соц}}}{100\%})}, 
\end{equation}

где $n$ -- категории работников, высвобождаемых в результате внедрения
программного средства (1);

    $\text{$\Delta$Ч}_{\text{i}}$ -- численность работников i-й категории, высвобожденных после
внедрения программного средства, (чел.); 
    
    $\text{З}_{\text{i}}$ -- годовая заработная плата высвобожденных работников i-й категории после внедрения программного средства, (р.);
    
    $\text{Н}_{\text{д}}$ -- норматив дополнительной заработной платы, (20\%); 
    
    $\text{Н}_{\text{соц}}$ -- ставка отчислений от заработной платы, включаемых в себестоимость, (34,6\%).
 
\begin{align*}
    \text{Э}_{\text{З}} &= 1 \cdot 37680 \cdot \left(1 + \frac{20\%}{100\%}\right) \cdot \left(1 + \frac{34,6\%}{100\%}\right) + \\
    &\quad + 1 \cdot 41280 \cdot \left(1 + \frac{20\%}{100\%}\right) \cdot \left(1 + \frac{34,6\%}{100\%}\right) + \\
    &\quad + 1 \cdot 30240 \cdot \left(1 + \frac{20\%}{100\%}\right) \cdot \left(1 + \frac{34,6\%}{100\%}\right) + \\
    &\quad + 1 \cdot 32400 \cdot \left(1 + \frac{20\%}{100\%}\right) \cdot \left(1 + \frac{34,6\%}{100\%}\right) \\
    &= 60860,74 + 66675,46 + 48843,65 + 52332,48 \\
    &= 228712,33 \text{ р}.
\end{align*}

Прирост чистой прибыли, полученной за счёт экономии на текущих затратах предприятия, будет рассчитываться по формуле:

\begin{equation}
    \text{$\Delta$П}_{\text{ч}} = (\text{Э}_{\text{тек}} - \text{$\Delta$З}_{\text{тек}})\cdot (1-\frac{\text{Н}_{\text{п}}}{100\%}), 
\end{equation}

где $\text{Э}_{\text{тек}}$ -- экономия на текущих затратах при использовании про-
граммного средства, (р.);

    $\text{$\Delta$З}_{\text{тек}}$ -- прирост текущих затрат, связанных с использованием ПО,
(р.);

    $\text{Н}_{\text{п}}$ -- ставка налога на прибыль, в соответствии с действующим законодательством, (18\%).


$$
    \text{$\Delta$П}_{\text{ч}} = (265345,07 - 15000)\cdot (1-\frac{18\%}{100\%}) = 205282,96 \text{ р}.
$$

\subsection{Расчет показателей экономической эффективности разработки и использования программного средства}

Оценка экономической эффективности разработки и использования программного средства для собственных нужд зависит от результата сравнения затрат на его разработку (модернизацию, совершенствование) и полученного экономического эффекта (годового прироста чистой прибыли).

1. Для организации-разработчика программного средства оценка экономической эффективности разработки осуществляется с помощью расчета простой нормы прибыли (рентабельности инвестиций (затрат) на разработку программного средства) по формуле:

\begin{equation}
    \text{Р}_{\text{и}} = \frac{\text{$\Delta$П}_{\text{ч}}}{\text{З}_{\text{р}}} \cdot 100\%,
\end{equation}

где $\text{$\Delta$П}_{\text{ч}}$ -- прирост чистой прибыли, полученной от разработки программного средства организацией-разработчиком по индивидуальному заказу, (р.);

$\text{З}_{\text{р}}$ -- затраты на разработку программного средства организацией-разработчиком,  (р.)


\begin{equation}
    \text{Р}_{\text{и}} = \frac{33918,49}{135673,96} \cdot 100\% = 25\%
\end{equation}

2. Поскольку сумма инвестиций меньше суммы годового экономического эффекта, то экономическая целесообразность инвестиций в разработку и использование программного продукта осуществляется на основе формулы: 

\begin{equation}
    \text{Р}_{\text{и}} = |\frac{\text{$\Delta$П}_{\text{ч}}}{\text{Ц}_{\text{п.с}} \cdot (1 + \frac{\text{Н}_{\text{д.с}}}{100\%})} - 1| \cdot 100\%,
\end{equation}

где $\text{Н}_{\text{д.с}}$ -- ставка налога на добавленную стоимость в соответствии с законодательством, {\%};

$\text{Ц}_{\text{п.с}}$ -- общая сумма затрат на разработку и реализацию программ-
ного средства, (р.).

$$
 \text{Ц}_{\text{п.с}} =\text{З}_{\text{о}} + \text{З}_{\text{д}} + \text{Р}_{\text{соц}} + \text{З}_{\text{пр}} + \text{Р}_{\text{р}}
$$

где $\text{Р}_{\text{р}}$ -- расходы на реализацию.

\begin{equation}
    \text{Р}_{\text{р}} = \frac{\text{З}_{\text{о}} \cdot \text{Н}_{\text{р}}}{100\%},
\end{equation}

где $\text{Н}_{\text{р}}$ -- норматив расходов на реализацию (5\%).

\begin{equation}
    \text{Р}_{\text{р}} = \frac{67325,31 \cdot 5\%}{100\%} = 3366,27 \text{ р}.
\end{equation}

Итого значение общих затрат на разработку и реализацию:

$$
 \text{Ц}_{\text{п.с}} = 67325,31+13465,06+27953,47+26930,12+3366,27 = 139040,23 \text{ р}.
$$


Тогда экономическая целесообразность инвестиций в разработку и использование программного продукта будет равна:

\begin{equation}
    \text{Р}_{\text{и}} = |\frac{205282, 96}{139040,23 \cdot (1 + \frac{0\%}{100\%})} - 1| \cdot 100\% = 47, 64 \%.
\end{equation}


\newpage

\subsection{Вывод}

В результате технико-экономического обоснования разработки интеллектуальной системы машинного перевода текстов на естественном языке были получены следующие значения показателей эффективности:

\begin{enumerate}
\item[а)] Рентабельность инвестиций организации-разработчика составляет
25 \%.
\item[б)] Затраты на разработку программного продукта окупятся.
\item[в)] Рентабельность инвестиций организации-заказчика составляет
47,64 \%.
\end{enumerate}

Таким образом, разработка и применение программного продукта является эффективной и данные инвестиции осуществлять целесообразно.