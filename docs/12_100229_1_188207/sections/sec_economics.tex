\section{Экономическое обоснование разработки платформы для интерактивного формирования запросов к языковым и генеративным нейросетям}
\label{sec:economics}

\subsection{Характеристика разработанного по индивидуальному заказу  программного средства}
Разрабатываемое программное средство представляет собой интеллектуальную платформу, предназначенную для интерактивного формирования запросов (промптов) к большим языковым и генеративным нейросетям. Иными словами, это инструмент для “prompt engineering”, помогающий пользователям правильно формулировать запросы к AI-моделям с целью получения релевантных и точных результатов.
Актуальность такого решения обусловлена бурным ростом применения генеративного ИИ в бизнесе: по данным опроса McKinsey, уже через год после появления массовых генеративных моделей треть компаний регулярно использует их как минимум в одной функции бизнеса​.
Руководители компаний все чаще лично работают с инструментами GPT-подобного ИИ​.
Однако эффективность этих моделей сильно зависит от качества запроса, поэтому умение правильно задавать вопрос нейросети стало критически важным навыком​.
Разрабатываемая платформа призвана упростить этот процесс для конечных пользователей.
Основные функции, которые выполняет программное средство:
\begin{enumerate}[label=\arabic*.]
	\item Разбивать запрос на логические блоки.
	\item Генерировать предпоказ результата использования запроса.
	\item Улучшать запрос.
	\item Расширять запрос.
	\item Загружать историю запросов.
	\item Перемещать слова в запросе как drag and drop элементы.
\end{enumerate}
Экономическая оценка целесообразности инвестиций в разработку и использование программного средства осуществляется на основе расчета и оценки следующих показателей: чистый дисконтированный доход, рентабельность инвестиций и срок окупаемости инвестиций

\subsection{Расчет затрат на 	 и цены программного средства по индивидуальному заказу}

Цена программного средства определена на основе полных затрат на разработку программного средства и включает в себя следующие статьи затрат:
\begin{itemize}
	\item затраты на основную заработную плату разработчиков;
	\item затраты на дополнительную заработную плату разработчиков;
	\item отчисления на социальные нужды;
	\item прочие затраты (амортизационные отчисления, расходы на электроэнергию, командировочные расходы, арендная плата за офисные помещения и оборудование, расходы на управление и реализацию и т.п.);
	\item общая сумма затрат на разработку;
	\item плановая прибыль, включаемая в цену программного средства;
	\item отпускная цена программного средства;
\end{itemize}

1. Затраты на основную заработную плату команды разработчиков.
Основная заработная плата исполнителей проекта определяется по формуле:

\begin{equation}
	\mbox{З}_{\mbox{о}}=\mbox{К}_{\mbox{пр}}\cdot \sum_{i=1}^{n}{\mbox{З}_{\mbox{ч}i}\cdot t_{i}},
\end{equation}

где	$n$ -- количество исполнителей, занятых разработкой конкретного ПО;

$\mbox{К}_{\mbox{пр}}$ -- коэффициент премий (1,5);

$\mbox{З}_{\mbox{ч}i}$ -- часовая заработная плата i-го исполнителя (руб.);

$t_{i}$ -- трудоемкость работ, выполняемых i-м исполнителем (ч).


Разработкой программного средства занимались следующие лица: бизнес–аналитик, 2 программиста, тестировщик, дизайнер, разработчик искусственного интеллекта. Часовая заработная плата каждого исполнителя определялась путем деления его месячной заработной платы (оклад) на количество рабочих часов в месяце.

Количество рабочих часов в месяце составляет 168.

Расчет основной заработной платы представлен в таблице \ref{tab1}.

\begin{table}[!h!t]
	\caption{Расчет основной заработной платы }
	\label{tab1}
	\centering

	\begin{tabular}{| >{\raggedright}m{0.02\textwidth}
		| >{\centering\arraybackslash}m{0.22\textwidth}
		| >{\centering\arraybackslash}m{0.15\textwidth}
		| >{\centering\arraybackslash}m{0.13\textwidth}
		| >{\centering\arraybackslash}m{0.2\textwidth}
		| >{\centering\arraybackslash}m{0.12\textwidth}|}

		\hline
		№                                                                                                                        & Участник команды                      & Месячный оклад, р. & Часовой оклад, р. & Трудоемкость работ, ч. & Итого, р. \\

		\hline
		1                                                                                                                        & 2                                     & 3                  & 4                 & 5                      & 6         \\

		\hline
		1                                                                                                                        & Бизнес–аналитик                       & 4000               & 23,80             & 80                     & 1904,00   \\

		\hline
		2                                                                                                                        & Программист                           & 2980               & 17,73             & 725                    & 12860,00  \\

		\hline
		3                                                                                                                        & Тестировщик                           & 1800               & 10,71             & 400                    & 4284,00   \\

		\hline
		4                                                                                                                        & Дизайнер                              & 3647               & 21,71             & 144                    & 3129,12   \\

		\hline
		5                                                                                                                        & Разработчик искусственного интеллекта & 3000               & 17,86             & 320                    & 5715,20   \\

		\hline
		\multicolumn{5}{|l|}{Итого}                                                                                              & 27892,32                                                                                                            \\

		\hline
		\multicolumn{5}{|l|}{Премия(50\%)}                                                                                       & 13946,16                                                                                                            \\
		\hline

		\multicolumn{5}{|l|}{\begin{tabular}[c]{@{}l@{}}Итого затраты на основную заработную плату\\ разработчиков\end{tabular}} & 41838,48                                                                                                            \\
		\hline
	\end{tabular}
\end{table}

2. Затраты на дополнительную заработную плату команды разработчиков включает выплаты, предусмотренные законодательством о труде (оплата трудовых отпусков, льготных часов, времени выполнения государственных обязанностей и других выплат, не связанных с основной деятельностью исполнителей), и определяется по формуле:

\begin{equation}
	\text{З}_{\text{д}} = \frac{\text{З}_{o}\cdot H_\text{д}}{100\%},
\end{equation}


где $H_\text{д}$ -- норматив дополнительной              заработной платы(20 \%);

$\text{З}_{\text{o}}$ -- затраты на основную заработную плату, (р.);




Дополнительная заработная плата составит:

$$
	\text{З}_{\text{д}} = \frac{41838,48 \cdot 20\%}{100\%} = 8367,70 \text{ р}.
$$

3. Отчисления в фонд социальной защиты и обязательного страхования (в фонд социальной защиты населения и на обязательное страхование) определяются в соответствии с действующими законодательными актами по формуле:

\begin{equation}
	\text{Р}_{\text{соц}} = \frac{(\text{З}_{o} + \text{З}_{\text{д}})\cdot H_\text{соц}}{100\%},
\end{equation}

где $H_\text{соц}$ -- норматив отчислений в фонд социальной защиты населения и на обязательное страхование (34,6 \%).

$$
	\text{Р}_{\text{соц}} = \frac{(41838,48 + 8367,70)\cdot 34,6\%}{100\%} = 17371,34 \text{ р}.
$$

4. Прочие затраты включаются в себестоимость разработки программного обеспечения в процентах от затрат на основную заработную плату команды разработчиков (табл.2.1) по формуле:

\begin{equation}
	\text{З}_{\text{пр}} = \frac{\text{З}_{o} \cdot \text{Н}_{\text{пр}}}{100},
\end{equation}

где $\text{Н}_{\text{пр}}$ -- норматив прочих затрат (40 \%).

$$
	\text{З}_{\text{пр}} = \frac{41838,48 \cdot 40\%}{100\%} = 16735,32 \text{ р}.
$$

5. Общая сумма затрат на разработку рассчитывается путем суммирования основной заработной платы,
дополнительной заработной платы, отчислений на социальные нужды, прочих затрат. Формула расчета имеет следующий вид:

$$
	\text{З}_{\text{р}} =\text{З}_{\text{о}} + \text{З}_{\text{д}} + \text{Р}_{\text{соц}} + \text{З}_{\text{пр}}
$$

$$
	\text{З}_{\text{р}} = 41838,48+8367,70+16735,32+17371,34 = 84312,84 \text{ р}.
$$

6. Плановая прибыль включает в себя два ключевых компонента: затраты на разработку и рентабельность этих затрат. Рентабельность затрат отражает желаемую прибыльность инвестиций и показывает, какой процент от затрат на разработку составит прибыль. Формула расчета имеет следующий вид:

\begin{equation}
	\text{П}_{\text{п.с}} = \frac{\text{З}_{\text{р}} \cdot \text{Р}_{\text{п.с}}}{100\%},
\end{equation}

где $\text{Р}_{\text{п.с}}$ -- рентабельность затрат на разработку программного средства (25 \%).

$$
	\text{П}_{\text{п.с}} = \frac{84312,84  \cdot 25\%}{100\%} = 21078,21 \text{ р}.
$$

7. Отпускная цена программного средства — это стоимость, по которой продукт предлагается заказчику. Она включает в себя все затраты на разработку, маркетинг и поддержку, а также предполагаемую прибыль. Отпускная цена может варьироваться в зависимости от рыночных условий, конкуренции и уникальных характеристик программного обеспечения. Формула расчета имеет следующий вид:

\begin{equation}
	\text{Ц}_{\text{п.с}} = \text{З}_{\text{р}} + \text{П}_{\text{п.с}}
\end{equation}

$$
	\text{Ц}_{\text{п.с}} = 84312,84 + 21078,21 = 105391,05 \text{ р}.
$$


Результаты расчета затрат на разработку представлены в таблице \ref{tab2}.

\begin{table}[!h!t]
	\centering
	\caption{Затраты на разработку программного обеспечения}
	\label{tab2}
	\begin{tabular}{|l|c|}
		\hline
		\multicolumn{1}{|c|}{Статья затрат}                                                                       & Сумма, руб. \\ \hline
		Основная заработная плата команды разработчиков                                                           & 41838,48    \\ \hline
		\begin{tabular}[c]{@{}l@{}}Дополнительная заработная плата команды\\ разработчиков\end{tabular}           & 8367,70     \\ \hline
		\begin{tabular}[c]{@{}l@{}}Отчисления в фонд социальной защиты и\\ обязательного страхования\end{tabular} & 17371,34    \\ \hline
		Прочие затраты                                                                                            & 16735,32    \\ \hline
		Общая сумма затрат на разработку                                                                          & 84312,84    \\ \hline
		\begin{tabular}[c]{@{}l@{}}Плановая прибыль, включаемая в цену \\программного средства\end{tabular}       &
		21078,21                                                                                                                \\ \hline
		Отпускная цена программного средства                                                                      &
		105391,05                                                                                                               \\ \hline
	\end{tabular}
\end{table}

\subsection{Расчет результата от разработки и использования программного средства по индивидуальному заказу}

Экономический эффект от разработки программного средства по индивидуальному заказу рассчитан для организации-разработчика (резидент Парка высоких технологий) и для организации-заказчика.

1. Для организации-разработчика экономическим эффектом является прирост чистой прибыли, полученной от разработки и реализации программного средства заказчику. Так как программное средство будет реализовываться организацией-разработчиком по отпускной цене, сформированной на основе затрат на разработку, то экономический эффект, полученный организацией-разработчиком, в виде прироста чистой прибыли от его разработки, определяется по формуле::

$$
	\text{$\Delta$П}_{\text{ч}} = \text{П}_{\text{п.с}} \cdot (1 - \frac{\text{Н}_{\text{п}} }{100\%}),
$$

где $\text{Н}_{\text{п}}$ -- ставка налога на прибыль,согласно действующему законодательству, (по состоянию на 01.01.2025 г. – 	20\%).;

$\text{П}_{\text{п.с}}$ -- прибыль, включаемая в цену программного средства, (р.);

$$
	\text{$\Delta$П}_{\text{ч}} = 21078,21 \cdot (1 - \frac{20\%}{100\%}) = 16862,57 \text{ р}.
$$

Исходя из расчетов, экономический эффект составляет 16862,57 \text{ р}.

Для организации-заказчика расчет экономического эффекта от применения программного обеспечения, разработанного по индивидуальному заказу сторонней организации, выполняется по следующей методике:

1. Экономия на заработной плате и начислениях на заработную плату сотрудников за счет снижения трудоемкости работ считается по формуле:


\begin{equation}
	\text{Э}_{\text{з.п}} = \text{К}_{\text{п.р}}\cdot (t^{\text{без п.с}}_{\text{р}}-t^{\text{с п.с}}_{\text{р}})\cdot \text{Т}_{\text{ч}}\cdot N_{\text{П}}\cdot  (1+\frac{\text{Н}_{\text{д}}}{100\%})\cdot (1+\frac{\text{Н}_{\text{соц}}}{100\%}),
\end{equation}

где $N_{\text{П}}$ -- плановый объем работ;

$t^{\text{без п.с}}_{\text{р}}-t^{\text{с п.с}}_{\text{р}}$ -- трудоемкость выполнения работы до и после внедрения программного продукта, нормо-час;

$\text{Т}_{\text{ч}}$ -- часовая тарифная ставка, соответствующая разряду выполняемых работ, руб./ч (15 руб./ч.);

$\text{К}_{\text{пр}}$ -- коэффициент премий, (1,5);

$\text{Н}_{\text{д}}$ -- норматив дополнительной заработной платы, (20\%);

$\text{Н}_{\text{соц}}$ -- ставка отчислений от заработной платы, включаемых в себестоимость, (34,6\%).

$$
	\text{Э}_{\text{З}} = 1,5 \cdot (8-4)\cdot 15\cdot 168 \cdot  (1+\frac{20\%}{100\%})\cdot (1+\frac{34,6\%}{100\%}) =  24421,82 \text{ р}.
$$
2. Экономия на заработной плате и начислениях на заработную плату в результате сокращения численности работников рассчитывается по формуле:

\begin{equation}
	\text{Э}_{\text{з.п}} = \sum_{i=1}^{n}{\text{$\Delta$Ч}_{\text{i}}\cdot \text{З}_{\text{i}}\cdot  (1+\frac{\text{Н}_{\text{д}}}{100\%})\cdot (1+\frac{\text{Н}_{\text{соц}}}{100\%})},
\end{equation}

где $n$ -- категории работников, высвобождаемых в результате внедрения
программного средства (1);

$\text{$\Delta$Ч}_{\text{i}}$ -- численность работников i-й категории, высвобожденных после
внедрения программного средства, (чел.);

$\text{З}_{\text{i}}$ -- годовая заработная плата высвобожденных работников i-й категории после внедрения программного средства, (р.);

$\text{Н}_{\text{д}}$ -- норматив дополнительной заработной платы, (20\%);

$\text{Н}_{\text{соц}}$ -- ставка отчислений от заработной платы, включаемых в себестоимость, (34,6\%).

\begin{align*}
	\text{Э}_{\text{З}} & = 1 \cdot 36000 \cdot \left(1 + \frac{20\%}{100\%}\right) \cdot \left(1 + \frac{34,6\%}{100\%}\right) +       \\
	                    & \quad + 1 \cdot 48000 \cdot \left(1 + \frac{20\%}{100\%}\right) \cdot \left(1 + \frac{34,6\%}{100\%}\right) + \\
	                    & \quad + 1 \cdot 21600 \cdot \left(1 + \frac{20\%}{100\%}\right) \cdot \left(1 + \frac{34,6\%}{100\%}\right) + \\
	                    & \quad + 1 \cdot 43764 \cdot \left(1 + \frac{20\%}{100\%}\right) \cdot \left(1 + \frac{34,6\%}{100\%}\right)   \\
	                    & = 58147,20 + 77529,60 + 34888,32 + 70687,61                                                                   \\
	                    & = 241252,73 \text{ р}.
\end{align*}

Прирост чистой прибыли, полученной за счёт экономии на текущих затратах предприятия, будет рассчитываться по формуле:

\begin{equation}
	\text{$\Delta$П}_{\text{ч}} = (\text{Э}_{\text{тек}} - \text{$\Delta$З}_{\text{тек}})\cdot (1-\frac{\text{Н}_{\text{п}}}{100\%}),
\end{equation}

где $\text{Э}_{\text{тек}}$ -- экономия на текущих затратах при использовании программного средства, (р.);

$\text{$\Delta$З}_{\text{тек}}$ -- прирост текущих затрат, связанных с использованием ПО,
(р.);

$\text{Н}_{\text{п}}$ -- ставка налога на прибыль, в соответствии с действующим законодательством, (20\%).


$$
	\text{$\Delta$П}_{\text{ч}} = (120560,21 - 60182,5)\cdot (1-\frac{20\%}{100\%}) = 48302,17 \text{р}.
$$

\subsection{Расчет показателей экономической эффективности разработки и использования программного средства}

Оценка экономической эффективности разработки и использования программного средства для собственных нужд зависит от результата сравнения затрат на его разработку (модернизацию, совершенствование) и полученного экономического эффекта (годового прироста чистой прибыли).

1. Для организации-разработчика программного средства оценка экономической эффективности разработки осуществляется с помощью расчета простой нормы прибыли (рентабельности инвестиций (затрат) на разработку программного средства) по формуле:

\begin{equation}
	\text{Р}_{\text{и}} = \frac{\text{$\Delta$П}_{\text{ч}}}{\text{З}_{\text{р}}} \cdot 100\%,
\end{equation}

где $\text{$\Delta$П}_{\text{ч}}$ -- прирост чистой прибыли, полученной от разработки программного средства организацией-разработчиком по индивидуальному заказу, (р.);

$\text{З}_{\text{р}}$ -- затраты на разработку программного средства организацией-разработчиком,  (р.)


\begin{equation}
	\text{Р}_{\text{и}} = \frac{21078,21}{84312,84} \cdot 100\% = 25\%
\end{equation}

2. Поскольку сумма инвестиций меньше суммы годового экономического эффекта, то экономическая целесообразность инвестиций в разработку и использование программного продукта осуществляется на основе формулы:

\begin{equation}
	\text{Р}_{\text{и}} = |\frac{\text{$\Delta$П}_{\text{ч}}}{\text{Ц}_{\text{п.с}} \cdot (1 + \frac{\text{Н}_{\text{д.с}}}{100\%})} - 1| \cdot 100\%,
\end{equation}

где $\text{Н}_{\text{д.с}}$ -- ставка налога на добавленную стоимость в соответствии с законодательством, {\%};

$\text{Ц}_{\text{п.с}}$ -- общая сумма затрат на разработку и реализацию программного средства, (р.).

$$
	\text{Ц}_{\text{п.с}} =\text{З}_{\text{о}} + \text{З}_{\text{д}} + \text{Р}_{\text{соц}} + \text{З}_{\text{пр}} + \text{Р}_{\text{р}}
$$

где $\text{Р}_{\text{р}}$ -- расходы на реализацию.

\begin{equation}
	\text{Р}_{\text{р}} = \frac{\text{З}_{\text{о}} \cdot \text{Н}_{\text{р}}}{100\%},
\end{equation}

где $\text{Н}_{\text{р}}$ -- норматив расходов на реализацию (5\%).

\begin{equation}
	\text{Р}_{\text{р}} = \frac{41838,48 \cdot 5\%}{100\%} = 2091,92 \text{ р}.
\end{equation}

Итого значение общих затрат на разработку и реализацию:

$$
	\text{Ц}_{\text{п.с}} = 41838,48+8367,70+17371,34+16735,32+2091,92 = 86404,76 \text{ р}.
$$


Тогда экономическая целесообразность инвестиций в разработку и использование программного продукта будет равна:

\begin{equation}
	\text{Р}_{\text{и}} = |\frac{16862,57 }{86404,76 \cdot (1 + \frac{25\%}{100\%})} - 1| \cdot 100\% = 84, 39 \%.
\end{equation}

Полученное значение экономической целесообразности показывает, какую чистую прибыль компания-разработчик может получить от вложений в разработку программного
обеспечения. Поскольку объём инвестиций превышает годовой прирост чистой прибыли, для организации-заказчика дополнительно рассчитываются
другие показатели экономической эффективности
Для приведения доходов и затрат к настоящему моменту времени
определяется коэффициент дисконтирования по формуле:

\begin{equation}
	\alpha_t = \frac{1}{\text{(1 + d)}^{\text{t}-\text{t}_\text{p}}}
\end{equation}
\newpage

где $d$ – требуемая норма дисконта, которая по своему смыслу соответствует устанавливаемому инвестором желаемому уровню рентабельности инвестиций, доли единицы;
$t$  – порядковый номер года, доходы и затраты которого приводятся к расчетному году;
$tp$  – расчетный год, к которому приводятся доходы и инвестиционные затраты.

Норму дисконта принимаем равным ставке рефинансирования Национального банка Республики Беларусь – 9,5\% \cite{refr}. Расчетный период составит четыре года.

\begin{equation}
	\begin{aligned}
		\alpha_1 & = \frac{1}{(1 + 0{,}095)^0} = 1{,}000      \\
		\alpha_2 & = \frac{1}{(1 + 0{,}095)^1} \approx 0{,}91 \\
		\alpha_3 & = \frac{1}{(1 + 0{,}095)^2} \approx 0{,}83 \\
		\alpha_4 & = \frac{1}{(1 + 0{,}095)^3} \approx 0{,}76
	\end{aligned}
\end{equation}

В первый год осуществляется разработка приложения, поэтому экономический эффект в этот период будет ниже ожидаемого.
Чтобы учесть это, необходимо определить продолжительность разработки. Поскольку команда работает параллельно поэтому берём за основу наибольшее количество часов у участника
проект и составляет 720 часов. Расчет эффективности инвестиций (затрат) в реализацию проектного решения представлен в таблице \ref{tab3}.
В данном случае дисконтированный эффект нарастающим итогом превысит дисконтированные инвестиции на третий год. Дисконтированный срок окупаемости рассчитывается по формуле:
\begin{equation}
	\text{T}_{\text{ок}} = \frac{\sum_{t=1}^{n}{\text{З}_t}}{\frac{1}{n} \cdot \sum_{t=1}^{n} \Delta\text{П}_{\text{чt}}},
\end{equation}

где $n$ – расчетный период, лет;

$\text{З}_t$ – затраты (инвестиции) в году t, р.;

$\Delta\text{П}_{\text{чt}}$  – прирост чистой прибыли в году t в результате реализации проекта, р.
\begin{table}[H]
	\caption{Расчёт эффективности инвестиций}
	\label{tab3}
	\centering
	\begin{tabular}{|p{6.3cm}|c|c|c|c|}
		\hline
		\multirow{2}{6.3cm}{\textbf{Показатель}}                 & \multicolumn{4}{c|}{\textbf{Годы расчётного периода}}                               \\
		\cline{2-5}
		                                                         & 1-й год                                               & 2-й год & 3-й год & 4-й год \\
		\hline
		1.\,Прирост чистой прибыли,~р.                           &
		36\,226,63                                                        &
		48\,302,17                                              &
		48\,302,17                                              &
		48\,302,17                                                                                                                                    \\
		\hline
		2.\,Дисконтированный результат,~р.                       &
		36\,226,63                                                        &
		43\,954,97                                              &
		40\,090,80                                             &
		36\,709,65                                                                                                                                    \\
		\hline
		3.\,Инвестиции в разработку,~р.                          &
		86\,404,76                                               &
		0                                                        &
		0                                                        &
		0                                                                                                                                              \\
		\hline
		4.\,Дисконтированные инвестиции,~р.                      &
		86\,404,76                                               &
		0                                                        &
		0                                                        &
		0                                                                                                                                              \\
		\hline
		5.\,Чистый дисконтированный доход по годам,~р.           &
		-50\,178,13                                              &
		-6\,223,16                                              &
		33\,867,64                                              &
		36\,709,65                                                                                                                                    \\
		\hline
		6.\,Чистый дисконтированный доход накопленным итогом,~р. &
		-50\,178,13                                              &
		-6\,223,16                                               &
		33\,867,64                                              &
		70\,577,29                                                                                                                                    \\
		\hline
		7.\,Коэффициент дисконтирования, доли ед.                &
		1,00                                                     &
		0,91                                                     &
		0,83                                                     &
		0,76                                                                                                                                           \\
		\hline
	\end{tabular}
\end{table}


Таким образом, дисконтированный срок окупаемости равен:

\begin{equation}
	\text{T}_{\text{ок}} = \frac{86\,404{,}76}{\frac{1}{4} \cdot (36\,226,63 + 48\,302,17 + 48\,302,17 + 48\,302,17)} = 1,91 \text{ года}.
\end{equation}
\begin{equation}
	\text{ИД}_{\text{PI}} = \frac{\sum_{t=1}^{n} \text{П}_{\text{чi}} \cdot \alpha_i}{\sum_{t=1}^{n} \text{З}_{i} \cdot \alpha_i}
\end{equation}

Таким образом, индекс доходности инвестиций равен:

\begin{equation}
	\text{ИД}_{\text{PI}} = \frac{36\,226,63 +43\,954,97 + 40\,090,80 + 36\,709,65}{86\,404,76} = 1,81.
\end{equation}

\newpage
\subsection{Вывод}

В результате технико-экономического обоснования разработкиа платформы для интерактивного формирования
запросов к языковым и генеративным нейросетям были получены следующие значения показателей эффективности:

\begin{enumerate}{locale=\arabic*.}
	\item Рентабельность инвестиций организации-разработчика составляет 25 \%.
	\item Затраты на разработку программного продукта окупятся.
	\item Рентабельность инвестиций организации-заказчика составляет 84,39 \%.
\end{enumerate}

Таким образом, разработка и применение программного продукта является эффективной и данные инвестиции осуществлять целесообразно.