\sectioncentered*{Введение}
\addcontentsline{toc}{section}{Введение}

За последние пять лет область \textit{prompt-engineering} и интерактивных
систем поддержки работы с крупными языковыми и генеративными нейронными
сетями (LLM / GDM) продемонстрировала существенные достижения.
С выходом моделей GPT-3.5 / GPT-4, Claude 2, LLaMA 2, DeepSeek-LLM и
аналогов стали доступны инструменты, обеспечивающие качество генерации
текста и изображений, сопоставимое с экспертным уровнем.  Появление
автоматических оптимизаторов запросов (AISEO Prompt Enhancer,
PromptPerfect), визуальных конструкторов для Midjourney, Stable Diffusion
и гибридных диалоговых ассистентов (OctiAI) вместе с методиками
chain-of-thought, self-consistency и few-shot‐обучения позволило резко
повысить точность и стабильность выводов моделей при решении прикладных
задач анализа текста, креативного письма и генерации мультимедиа.
Актуальность разработки платформ, упрощающих формирование и
оптимизацию запросов, обусловлена постоянным ростом числа конечных
пользователей, не обладающих глубокими знаниями в области ИИ,
одновременно предъявляющих повышенные требования к
надёжности и повторяемости результатов.

\bigskip
\textbf{Цель дипломного проектирования} — разработка
веб-платформы, обеспечивающей \emph{интерактивное формирование,
структурирование и оптимизацию запросов} к языковым и генеративным
нейросетям с возможностью последующей интеграции в корпоративные и образовательные процессы.

\bigskip
\textbf{В основу проектирования} положены следующие принципы:
\begin{itemize}
  \item \emph{Модульность и микросервисная архитектура} —
        разделение подсистем (клиент Vue 3, сервер FastAPI)
         с независимым масштабированием;
  \item \emph{Непрерывная интеграция/доставка (CI/CD)} и контроль
        качества кода средствами GitHub Actions и SonarQube;
  \item \emph{Эвристики prompt-engineering} — использование
        структурных делимитеров, chain-of-thought-шаблонов,
        автоматического рефрейзинга и скоринговых метрик качества
        запроса;
  \item \emph{Пользовательско-центричный дизайн} —
        адаптивный интерфейс drag-and-drop, поддержка многоязычия и
        доступность WCAG 2.1 AA;
  \item \emph{Информационная безопасность и конфиденциальность} —
        OWASP Top-10 hardening, RBAC, шифрование данных в покое
        и транзите.
\end{itemize}

\bigskip
\textbf{Структура пояснительной записки} отражает последовательное
решение поставленных задач:

\begin{description}
  \item[1 Анализ подходов и технологий.]
        Выполняется сравнительный анализ современных методик
        улучшения детализированности и структуры запросов,
        рассматриваются существующие платформы, проводится
        классификация архитектурных вариантов и выбор целевых
        инструментов разработки.  \emph{Задача раздела —
        обоснование технологической и методической базы проекта}.
  \item[2 Проектирование модели платформы.]
        Определяются целевая аудитория, функциональные требования,
        информационные потоки и архитектура; формализуются алгоритмы
        оптимизации запросов, схема БД PostgreSQL + pgvector и
        модели безопасности.  \emph{Задача раздела —
        разработка логической и физической модели системы}.
  \item[3 Реализация программных компонентов.]
        Описываются детали клиентского приложения (Vue 3 + Pinia),
        серверного API (FastAPI), ML-сервисов, интеграции с FusionBrain
        и DeepSeek, а также методика оценки качества сформированных
        запросов.  \emph{Задача раздела — демонстрация
        работоспособности и экспериментальная валидация}.
  \item[4 Технико-экономическое обоснование.]
        Рассчитываются затраты на разработку и эксплуатацию,
        экономический эффект от внедрения и показатели
        эффективности проекта (NPV, ROI, PI).  \emph{Задача раздела —
        подтверждение экономической целесообразности}.
  \item[Заключение.]
        Подводятся итоги исследования, формулируются выводы и
        направления дальнейшего развития платформы.
\end{description}

\bigskip
Дипломный проект выполнен \emph{самостоятельно}, проверен системой
«Антиплагиат»; процент оригинальности соответствует нормам кафедры ИИТ.
Все заимствования оформлены ссылками на публикации, приведённые
в~«Списке использованных источников».
