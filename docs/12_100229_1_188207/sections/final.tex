\sectioncentered*{Заключение}
\addcontentsline{toc}{section}{Заключение}

В ходе выполнения дипломной работы изучены существующие методы интерактивного формирования запросов к языковым и генеративным нейросетям, а также исследованы современные подходы к интеграции с внешними нейросетевыми сервисами. На основании проведенного анализа сформулированы функциональные требования к разрабатываемой платформе. Также предложена её архитектура в виде клиент-серверного приложения с разделением на клиентскую и серверную части (Frontend + Backend).

В соответствии с сформулированными требованиями разработана программная платформа с архитектурой «клиент-сервер»: клиентская часть реализована на базе JavaScript-фреймворка Vue.js, серверная — на базе Python-фреймворка FastAPI. Пользовательский интерфейс системы обеспечивает визуальное конструирование запросов методом «drag-and-drop». Для хранения данных о запросах и результатах используется база данных PostgreSQL с модулем pgvector, позволяющим сохранять векторные представления данных и выполнять по ним поиск. Реализована интеграция с внешними API: FusionBrain (для генерации изображений) и DeepSeek (для обработки естественного языка). Алгоритмы обработки пользовательских запросов на стороне сервера автоматически преобразуют сформированные визуальные схемы в последовательность вызовов нейросетевых моделей.

Разработанная платформа испытана в различных сценариях использования, подтвердивших корректность её работы и соответствие предъявленным требованиям. Экспериментально показано, что предложенное решение эффективно автоматизирует процесс формирования сложных запросов к языковым и генеративным моделям, повышая удобство и эффективность использования нейросетей. Также выполнено технико-экономическое обоснование проекта. Анализ показал, что применение современных открытых инструментов (Vue, FastAPI, PostgreSQL) и внешних API (FusionBrain, DeepSeek) обеспечивает высокую функциональность системы при оптимальных затратах, подтверждая целесообразность её внедрения. Таким образом, цель дипломной работы достигнута: разработана и исследована платформа для интерактивного формирования запросов к языковым и генеративным нейросетям, продемонстрировавшая практическую применимость и эффективность предложенного подхода.