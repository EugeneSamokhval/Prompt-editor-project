\sectioncentered*{Перечень условных обозначений} \addcontentsline{toc}{section}{Перечень условных обозначений} \label{sec:reduction}


\par \textbf{БД} – база данных;
\par \textbf{ГОСТ} – Государственный стандарт;
\par \textbf{ООО} – Общество с ограниченной ответственностью;
\par \textbf{ОЗУ} – оперативная память;
\par \textbf{ОС} – операционная система;
\par \textbf{ПК} – персональный компьютер;
\par \textbf{СУБД} – система управления базами данных.
\par \textbf{СТБ} – Стандарт Республики Беларусь;
\par \textbf{СУОТ} – система управления охраной труда;
\par \textbf{ЗАО} – Закрытое акционерное общество;
\par \textbf{ТНПА} – технические нормативно-правовые акты.

\par \textbf{ACID} – Atomicity, Consistency, Isolation, Durability (атомарность, согласованность, изоляция, долговечность);
\par \textbf{AI} – Artificial Intelligence (искусственный интеллект);
\par \textbf{APE} – Automatic Prompt Engineer (автоматический инженер запросов);
\par \textbf{API} – Application Programming Interface (программный интерфейс приложений);
\par \textbf{ASGI} – Asynchronous Server Gateway Interface (асинхронный шлюзовый интерфейс сервера);
\par \textbf{AWS} – Amazon Web Services (облачная платформа Amazon);
\par \textbf{CDN} – Content Delivery Network (сеть доставки контента);
\par \textbf{CI} – Continuous Integration (непрерывная интеграция);
 \par \textbf{CI/CD} – Continuous Integration / Continuous Delivery (непрерывная интеграция / непрерывная доставка);
\par \textbf{CPU} – Central Processing Unit (центральный процессор);
\par \textbf{CUDA} – Compute Unified Device Architecture (универсальная архитектура вычислений наGPU отNVIDIA);
\par \textbf{DB} – DataBase (база данных);
\par \textbf{DevOps} – Development \& Operations (комплекс практик для объединения процессов разработки и эксплуатации);
\par \textbf{DoS} – Denial of Service (отказ в обслуживании);
\par \textbf{FaaS} – Functions as a Service (функции как сервис);
\par \textbf{GPT} – Generative Pre-trained Transformer (генеративно-дополнительно обученная модель-трансформер);
\par \textbf{GPU} – Graphics Processing Unit (графический процессор);
\par \textbf{HTML} – HyperText Markup Language (язык гипертекстовой разметки);
\par \textbf{HTTP} – HyperText Transfer Protocol (протокол передачи гипертекста);
\par \textbf{HTTPS} – HyperText Transfer Protocol Secure (защищённый протокол передачи гипертекста);
\par \textbf{IEEE} – Institute of Electrical and Electronics Engineers (Институт инженеров по электротехнике и электронике);
\par \textbf{ISO} – International Organization for Standardization (Международная организация по стандартизации);
\par \textbf{IT} – Information Technology (информационные технологии);
\par \textbf{JWT} – JSON Web Token (JSON-веб-токен);
\par \textbf{LLaMA} – Large Language Model Meta AI (семейство больших языковых моделей от Meta AI);
\par \textbf{LLM} – Large Language Model (большая языковая модель);
\par \textbf{ML} – Machine Learning (машинное обучение);
\par \textbf{NLP} – Natural Language Processing (обработка естественного языка);
\par \textbf{ORM} – Object-Relational Mapping (объектно-реляционное отображение);
\par \textbf{OWASP} – Open Web Application Security Project (проект по веб-безопасности);
\par \textbf{RAM} – Random Access Memory (оперативная память);
\par \textbf{RBAC} – Role-Based Access Control (управление доступом на основе ролей);
\par \textbf{REST} – Representational State Transfer (архитектурный стиль взаимодействия клиент-сервер);
\par \textbf{RL} – Reinforcement Learning (обучение с подкреплением);
\par \textbf{SPA} – Single Page Application (одностраничное веб-приложение);
\par \textbf{SQL} – Structured Query Language (язык структурированных запросов);
\par \textbf{SSH} – Secure Shell (безопасная оболочка удалённого доступа);
\par \textbf{SSL} – Secure Sockets Layer (уровень защищённых сокетов);
\par \textbf{T5} – Text-to-Text Transfer Transformer (модель для преобразования текста);
\par \textbf{UI} – User Interface (пользовательский интерфейс);
\par \textbf{VM} – Virtual Machine (виртуальная машина);
\par \textbf{VRAM} – Video Random Access Memory (видеопамять).