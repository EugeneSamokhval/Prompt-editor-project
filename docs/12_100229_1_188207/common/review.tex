% Содержимое данного документа позаимсвовано из Приложения Ж из документа https://library.bsuir.by/m/12_101945_1_141950.pdf
\documentclass[12pt,a4paper]{article}
\usepackage[utf8]{inputenc}
\usepackage[russian]{babel}
\usepackage[T2A]{fontenc}
% ---► добавьте ЭТО:                                  
\usepackage[
  a4paper,            % формат бумаги
  left=15mm,          % левое поле
  right=15mm,         % правое поле
  top=20mm,           % верхнее поле
  bottom=20mm,        % нижнее поле
]{geometry}
% ◄---
\begin{document}

\thispagestyle{empty}

%\newcommand{\fio}{Иванов Иван Иванович}
\newcommand{\topicName}{Система передачи данных}

{\small
  \begin{center}
    \begin{minipage}{0.9\textwidth}
      \begin{center}
        {\normalsize РЕЦЕНЗИЯ}\\[0.2cm]
на дипломный проект студента факультета информационных технологий и управления
учреждения образования «Белорусский государственный университет информатики и радиоэлектроники»
        %\fio \\
        Самохвала Евгения Сергеевича\\ на тему: <<Платформа для интерактивного формирования запросов к языковым и генеративным нейросетям>>
      \end{center}
    \end{minipage}\\
  \end{center}

Дипломный проект студента Самохвала Евгения Сергеевича состоит из 6 листов графического материала формата А1 и 79 страниц пояснительной записки.

Тема проекта является актуальной и посвящена разработке платформы для интерактивного формирования запросов к языковым и генеративным нейросетям. Разработка подобной платформы обусловлена стремительным развитием технологий искусственного интеллекта и необходимостью упрощения взаимодействия пользователя с большими языковыми моделями.

Пояснительная записка построена логично и последовательно, отражает все этапы разработки в соответствии с календарным планом. В пояснительной записке достаточно полно выполнен обзор современных систем, проанализированы методы оптимизации запросов к большим языковым моделям (LLM).

В дипломном проекте разработана микросервисная архитектура платформы; реализована интеграция с современными языковыми моделями FusionBrain и DeepSeek; программная часть выполнена с использованием современного серверного фреймворка FastAPI и фронтенд-фреймворка Vue 3. В проекте приведен глубокий аналитический обзор научно-технической литературы, касающиеся темы проекта. Выполнено технико-экономическое обоснование проекта: рассчитаны показатели NPV, ROI, PI, подтверждающие экономическую целесообразность разработки.

Приведенные в дипломном проекте решения и программная реализация свидетельствуют о глубоких знаниях студента Самохвала Е. С. в области разработки подобных систем, умении работать с технической литературой и применять на практике наиболее рациональные решения. По каждому разделу и в целом по дипломному проекту приведены аргументированные выводы.

Пояснительная записка и графический материал оформлены аккуратно и в соответствии с установленными требованиями. Считаю, что разработанная платформа и полученные результаты могут быть применены на практике для повышения эффективности взаимодействия с языковыми и генеративными моделями.
\\Замечания:
\begin{itemize}
  \item интерфейс платформы обеспечивает доступ ко всему функционалу, однако с точки зрения эстетики и удобства восприятия требует дополнительных улучшений;
  \item не реализована часть стратегий по улучшению запросов к языковым нейросетям, описанных в первом разделе пояснительной записки;
  \item в пояснительной записке имеются отдельные незначительные опечатки и недочеты оформления (например, в некоторых местах нарушена единообразная нумерация рисунков);
  \item отдельные формулировки изложены недостаточно конкретно, что допускает двоякое толкование некоторых положений.
\end{itemize}

В целом дипломный проект выполнен технически грамотно, в полном соответствии с техническим заданием на проектирование и заслуживает оценки 9 баллов, а дипломник Самохвал Е. С. – присвоения квалификации <<Инженер-системотехник>>.

  \vfill
  \noindent
  \begin{minipage}{0.4\textwidth}
    \begin{flushleft}
      Рецензент:\\
      ассистент\\
      кафедры электронных вычислительных машин, БГУИР\\
      \underline{\hspace*{2em}} \underline{\hspace*{6.5em}} \the\year{}~г.
    \end{flushleft}
  \end{minipage}
  \begin{minipage}{0.58\textwidth}
    \begin{flushright}
    \underline{\hspace*{3cm}}\hspace*{0.5cm} С.\,С.~Силич \\
    \end{flushright}
  \end{minipage}
}

\clearpage
\end{document}