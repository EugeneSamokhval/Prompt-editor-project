% Содержимое данного документа позаимсвовано из Приложения Е из документа https://library.bsuir.by/m/12_101945_1_141950.pdf

% Содержимое данного документа позаимсвовано из Приложения Е из документа http://www.bsuir.by/m/12_113415_1_66883.pdf
\documentclass[12pt,a4paper]{article}
\usepackage[utf8]{inputenc}
\usepackage[russian]{babel}
\usepackage[T2A]{fontenc}
\setlength{\parindent}{1.25cm}
\begin{document}

\thispagestyle{empty}

\begin{singlespace}
{\small
  \begin{center}
    \begin{minipage}{0.8\textwidth}
      \begin{center}
        {\normalsize ОТЗЫВ}\\[1em]
        на дипломный проект\\студента факультета информационных технологий 
        и управления\\Учреждения образования <<Белорусский государственный университет информатики и радиоэлектроники>>\\
        %\fio \\
        Самохвала Евгения Сергеевича\\
        на тему: <<Платформа для интерактивного формирования запросов к языковым и генеративным нейросетям>>
      \end{center}
    \end{minipage}
  \end{center}

Дипломный проект включает разработку оригинальной программной платформы, предназначенной для интерактивного формирования запросов к современным языковым и генеративным нейронным сетям. В рамках проекта студентом спроектирована собственная архитектура системы, реализована клиентская часть на фреймворке Vue.js и серверная логика на базе технологии FastAPI. В разработанную платформу интегрированы внешние API сервисов искусственного интеллекта FusionBrain и DeepSeek, что позволило обеспечить взаимодействие с современными языковыми моделями и значительно расширить функциональность системы.

Важным достижением проекта является реализация механизма векторного поиска, позволяющего эффективно обрабатывать и сопоставлять текстовые запросы и результаты генерируемые нейросетями. Кроме того, автором проекта проведён расчёт экономической эффективности внедрения разработанной платформы, подтвердивший её практическую ценность и экономическую целесообразность. Пояснительная записка и техническая документация к проекту выполнены аккуратно и в полном соответствии с требованиями ЕСКД, что свидетельствует о внимательности автора к деталям оформления.

Дипломная работа полностью соответствует выданному техническому заданию: все поставленные задачи выполнены, а заявленные цели достигнуты. Проект отличается оригинальностью предложенного решения и комплексным подходом к поставленной задаче. Автор продемонстрировал умение сочетать современные инструменты веб-разработки с технологиями искусственного интеллекта, и полученные результаты свидетельствуют о высоком уровне подготовки студента и его готовности к самостоятельной профессиональной деятельности по специальности.

Считаю, что дипломный проект Самохвала Е.~С. выполнен на высоком уровне, а его автор заслуживает присвоения квалификации «Инженер-системотехник». Рекомендую допустить студента к защите дипломной работы и присвоить ему указанную квалификацию по специальности.\\
\vfill
  \noindent
  \begin{minipage}{0.54\textwidth}
    \begin{flushleft}
      Руководитель дипломного проекта:\\
      канд. техн. наук,\\ доцент кафедры ИИТ, БГУИР \\
      \\
      \underline{\hspace*{2em}} \underline{\hspace*{6.5em}} \the\year{}~г.
    \end{flushleft}
  \end{minipage}
  \begin{minipage}{0.44\textwidth}
    \begin{flushright}
      \underline{\hspace*{3cm}} Ю.\,Б.~Крапивин
    \end{flushright}
  \end{minipage}
}

\end{singlespace}


\clearpage
\end{document}