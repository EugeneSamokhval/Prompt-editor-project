\sectioncentered*{Заключение}
\addcontentsline{toc}{section}{Заключение}
\label{sec:practice:conclusion}
В ходе дипломного проектирования была спроектирована и частично реализована платформа для интерактивного формирования, оценки и предпросмотра запросов (промптов) к языковым и генеративным нейросетям. Основными целями разработки стали удобство пользователя, сокращение числа итераций при подборе формулировок и возможность быстрой адаптации промпта под конкретную модель. Проект охватывает все этапы работы с запросом: от начального ввода текста или набора ключевых слов до автоматического дополнения деталями, выставления оценки качества и демонстрации предварительного результата – текста, сгенерированного локальной моделью LLaMA 8B, либо изображения, полученного через API Fusion Brain (модель Kandinsky 3.1).

В основе архитектуры лежит классический клиент-серверный подход. Серверная часть реализована на фреймворке FastAPI (Python), что обеспечивает высокую производительность и удобную интеграцию с нейросетями. Для хранения данных и истории запросов используется база PostgreSQL, а интерфейс строится как одностраничное приложение (SPA) на Vue.js, давая пользователям наглядный и отзывчивый UI с возможностью переключаться между текстовым и графическим режимами редактирования промпта. Проект ориентирован как на разработчиков, так и на художников: в нём совмещаются гибкие инструменты тонкой настройки запросов и дружественная среда для творческих специалистов. Реализованные в ходе работы технические решения подтверждают достижимость заложенных требований: система способна ускорить и упростить процесс составления эффективных промптов, позволяя быстро проверять и улучшать результат. Дальнейшая реализация и тестирование платформы покажут её практические преимущества и дадут возможность масштабировать проект для более широкого круга пользователей.